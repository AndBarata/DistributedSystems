\documentclass[conference]{IEEEtran}
\IEEEoverridecommandlockouts
% The preceding line is only needed to identify funding in the first footnote. If that is unneeded, please comment it out.
\usepackage{cite}
\usepackage{amsmath,amssymb,amsfonts}
\usepackage{algorithmic}
\usepackage{graphicx}
\usepackage{textcomp}
\usepackage{xcolor}
\usepackage{url}
\usepackage{hyperref} 
\usepackage{amsmath}

\def\BibTeX{{\rm B\kern-.05em{\sc i\kern-.025em b}\kern-.08em
    T\kern-.1667em\lower.7ex\hbox{E}\kern-.125emX}}
\begin{document}

\title{Sincronização de semáforos baseado em NTP\\}

\author{\IEEEauthorblockN{1\textsuperscript{st} André de Azevedo Barata}
\IEEEauthorblockA{
up201907705@up.pt}
\and
\IEEEauthorblockN{2\textsuperscript{nd} Diogo Vilela}
\IEEEauthorblockA{
up201907804@up.pt}

}

\maketitle

\begin{abstract}

\end{abstract}


\section{Introdução}


    Este trabalho apresenta um caso prático de sincronização de relógios usando o Network Time Protocol (NTP). O problema apresentado consiste em sincronizar as transições de estado de 2 semáforos de uma interseção sem que haja trocas de informações entre os eles. Deste modo, todos os semáforos irão utilizar um relógio abstrato, escravizado de acordo com um servidor NTP, para inferir o seu estado (Verde, Vermelho).
    
    Por exemplo, considerando a figura \ref{fig:diagramaEstrada}, quando o semáforo da via horizontal (1) está verde, o da via vertical (2) está vermelho. Ora, caso esta alteração de estado ocorra de $t$ em $t$ segundos, cada semáforo deve tomar essa decisão autonomamente de acordo com o seu relógio.
    Os dois relógios apresentam características diferentes, pelo que $t$ segundos no semáforo 1 pode representar $t + \delta$ segundos no relógio 2, pelo que a transição no 2 seria tomada $\delta$ segundos mais tarde. Normalmente, $\delta$ é um valor muito pequeno, pelo que quando considerando uma só transição este valor é irrelevante. Contudo, esta discrepância acumula com o tempo e, sem a aplicação de mecanismos de sincronização, pode levar ao caso de os dois semáforos estarem verdes ao mesmo tempo.
    Uma correta sincronização não irá colocar $\delta$ a zero, mas sim mantê-lo com um valor baixo o suficiente para que seja irrelevante para o sistema e constante ao longo do tempo. Para tal, cada semáforo irá atualizar periodicamente o "rate", o "offset" e o "delay" dos seus relógios com um servidor NTP, garantindo que $\delta$ não acumula infinitamente.  
    
    \begin{figure}[h]
        \centering
        \includegraphics[width=0.8\linewidth]{figures/diagramaEstrada.png}
        \caption{Interseção de uma estrada}
        \label{fig:diagramaEstrada}
    \end{figure}

    Atualmente, as soluções para a mudança de estado dos semáforos pode ser agrupada em 2 grupos distintos, um primeiro baseado em sistemas de tempo fixo e um segundo em sistemas baseados em sensores. Para o primeiro caso, a mudança de estado é feita através de ciclos de tempo fixo, com base no relógio de cada semáforo. Contudo, maior parte dos semáforos não têm qualquer tipo de sincronização, tendo, em vez disso uma manutenção periódica aos seus relógios. O segundo caso utiliza sensores para medir o fluxo de trânsito, ou controlar a velocidade. Com base nos valores medidos por estes sensores, uma decisão é posteriormente tomada pelo sistema.
    
    
\section{Protocolo NTP}
\label{sec:NTP}

O Network Time Protocol é um protocolo de sincronização
de relógios que sincroniza os relógios de um sistema distribuído 
através de redes com comutação de pacotes e de latência variável 
com uma precisão na ordem dos milissegundos. Este protocolo
foi primordialmente concebido para ter uma alta exatidão 
e fiabilidade \cite{b2}.

Para uma típica operação do protocolo NTP,
o cliente questiona um ou mais servidores NTP, de modo a 
receber o tempo atualizado. Após a receção dos dados do servidor, 
o cliente calcula o offset do seu relógio relativamente ao servidor, o delay da rede e o rate. Para tal, é necessário q ele saiba os timestamps da mensagem enviada por ele ao servidor e da consequente resposta do servidor,
para tal figura \ref{fig:diagramaNTP} contextualiza o protocolo de melhor
forma. Os timestamps que são necessário são então o tempo a que
a mensagem foi enviada pelo cliente, t0, e quando foi recebida pelo
servidor, t1, tal como quando a mensagem de resposta foi enviada
pelo servidor, t2, e finalmente recebida pelo cliente, t3.

    \begin{figure}[h]
        \centering
        \includegraphics[width=0.8\linewidth]{figures/diagramaNTP.png}
        \caption{Ordem de eventos do protocolo NTP}
        \label{fig:diagramaNTP}
    \end{figure}

Com estes 4 tempos é então possível calcular quer o valor do offset quer o valor 
do delay da rede. O offset, $\theta$, vai corresponder 
à diferença entre o relógio do cliente e do servidor, sendo o valor dado pela equação \ref{eq:offset}. 


\begin{equation} \label{eq:offset}
\theta = \frac{\left( (t_{\text{1}} - t_{\text{0}}) + (t_{\text{2}} - t_{\text{3}}) \right)}{2} 
\end{equation}

O delay da rede, $\delta$, corresponde por sua vez ao tempo
que um pacote demora a ir do cliente ao servidor e vice-versa, sendo
o valor dado pela equação \ref{eq:delay}.


\begin{equation} \label{eq:delay}
\delta = (t_{\text{3}} - t_{\text{0}}) - (t_{\text{2}} - t_{\text{1}})
\end{equation}

Para além disso também pode ser calculado o rate, que serve
para compensar o desvio do relógio do cliente quando comparado com o do servidor.
Para tal, para o sistema criado no âmbito do projeto definiu-se a função do
rate de acordo com a equação \ref{eq:rate}. Na equação apresentada,
o valor $t_{\text{3}}'$ e $t_{\text{1}}'$ correspondem aos timestamps 
mais recentes, enquanto t1 e t3 correspondem aos timestamps da mensagem anterior à
atual. Assim, o rate vai sendo modificado com base nos timestamps
das duas últimas iterações entre cliente-servidor.


\begin{equation} \label{eq:rate}
\text{{rate}} =\left| \frac{{t_{\text{1}}' - t_{\text{1}} - \delta + \delta '}}{{t_{\text{3}}' - t_{\text{3}}}} \right|
\end{equation}


Por fim, a equação \ref{eq:corrected_time} foi concebida, de modo a corrigir
o tempo do relógio do cliente, sendo a variável \text{{elapsed\_time}} o tempo
que entre $t_{\text{3}}$ e o tempo em que a correção do tempo foi efetuada.


\begin{equation} \label{eq:corrected_time}
\text{{corrected\_time}} = t_{\text{3}} + \text{{elapsed\_time}} \cdot \text{{rate}} + \theta
\end{equation}



\section{Arquitetura do Sistema}

    Na interseção descrita na figura \ref{fig:diagramaEstrada} existem 4 semáforos, pelo que, idealmente, cada um seria representado por uma raspberry Pi. Contudo, devido ao facto de apenas existirem duas raspberrypis disponíveis, os dois semáforos de cada sentido serão controlados por  uma das duas raspberrys e terão o mesmo comportamento. Assim, quando a raspberry Pi A está no estado VERDE é permitido o transito horizontal e quando a raspberry Pi B está VERDE é permitido o trânsito vertical. As raspberrypis nunca apresentam o mesmo estado em simultâneo. 

    O sistema foi desenhado de acordo com a figura \ref{fig:diagramaSistema}. Ambas as raspberrypis foram conectadas à rede Wi-Fi do portátil, que por sua vez comunica com o servidor NTP \url{pool.ntp.org}. Cada raspberry Pi representa um cliente que envia para o portátil os pedidos NTP de acordo com a secção \ref{sec:NTP}, que por sua vez são reencaminhados para o servidor. A resposta faz o caminho oposto. O estado da cada raspberry Pi é enviado para o portátil cada vez que existe uma transição. Este processo serve apenas para monitorização e está representado na figura como monitor. O monitor regista o tempo local da chegada da mensagem de mudança de estado de cada raspberry Pi e guarda o valor num ficheiro .txt. Por exemplo, A fica VERMELHO e B fica VERDE, o monitor mostra o tempo de cada transição (que deve ser muito semelhante).

    O protocolo empregue para a comunicação é o "Transmission Control Protocol" - TCP. Se o servidor NTP não responder à solicitação, é acionado um "timeout", e o sistema tenta restabelecer a ligação. Esta funcionalidade permite que a sincronização continue sem que o servidor esteja disponível, utilizando os último valores registados.

    \begin{figure}[h]
        \centering
        \includegraphics[width=0.8\linewidth]{figures/diagramaSistema.png}
        \caption{Diagrama da arquitetura do sistema \cite{b1}}
        \label{fig:diagramaSistema}
    \end{figure}
\section{Implementação} \label{sec:implementação}
    Durante esta secção, irá ser descrito o algoritmo de sincronização e posteriormente a sua implementação em Python. Para além disso, é feita a distinção entre três relógios:
    
    \begin{itemize}
        \item \textbf{Relógio monotónico}:  Relógio intrínseco do hardware. Nunca é corrigido e conta o tempo a partir do qual o sistema operativo foi iniciado.
        \item \textbf{Relógio NTP}: Relógio do servidor NTP e assumido como correto. Este relógio escraviza o sistema.
        \item \textbf{Relógio abstrato}: Relógio implementado em software em cima do sistema operativo. É o relógio monotónico corrigido em "rate" e "offset" de acordo com a secção \ref{sec:NTP}.
    \end{itemize}

    Posto isto, as raspberrypis têm dois relógios: monotónico e abstrato, sendo que este último é utilizado para tomar as decisões de mudança de estado. O monitor apenas utiliza o seu relógio monotónico para marcar os "timestamps" das mudanças de estado de A e B. O servidor NTP responde aos pedido com os "timestamps" do relógio NTP. 

\subsection{Algoritmo}
    O programa está dividido em dois processos distintos que ocorrem em paralelo: Correção do relógio e Atualizado do Estado. Ambos os algoritmos estão expressos em pseudo-código nas figuras \ref{fig:pseudoCodigoRelogio} e \ref{fig:pseudoCodigoEstado}, respetivamente.
    
    

    \begin{figure}[h]
        \centering
        \includegraphics[width=0.6\linewidth]{figures/pseudoCodigoRelogio.png}
        \caption{Pseudo-código relativo à correção do relógio abstrato}
        \label{fig:pseudoCodigoRelogio}
    \end{figure}

    De referir que, o relógio abstrato é o relógio corrigido segundo os parâmetros obtido pelo último pedido NTP (offset, rate e delay), segundo as equações da secção \ref{sec:NTP}. A consulta do relógio abstrato consiste em calcular o tempo que passou desde da última correção NTP de acordo com o relógio monotónico, multiplicar esse tempo pelo rate, somar ao último timestamp desde a atualização e adicionar o offset, equação \ref{eq:corrected_time}.


    
    \begin{figure}[h]
        \centering
        \includegraphics[width=0.6\linewidth]{figures/pseudoCodigoEstado.png}
        \caption{Pseudo-código relativo à atualização do estado}
        \label{fig:pseudoCodigoEstado}
    \end{figure}


\subsection{Implementação em Python}

\textcolor{red}{\textbf{Breve menção ao código, threads e class e referência ao git}
\section{Resultados} \label{sec:resultados}

\textcolor{red}{\textbf{Experiências em table: \\
    - Correção offset e rate\\
    - Correção só offset\\
    - Sem correcção\\
Resultados: \\
    - Erro nas slots \\
    - min, max, variação do rate, delay, offset}}
\section{Conclusão} \label{sec_conclusão}

\subsection{Análise dos resultados}
De acordo com os resultados da tabela \ref{tab:resultados} é possível inferir o seguinte:

\begin{enumerate}
    \item O servidor mais estável é o S3: devido aos valor baixos do jitter, algo esperado, visto que é um servidor local.
    \item O experiência que apresentou melhores resultados foi quando não houve correção de relógio.
    \item A experiências em que o jitter é mais baixo apresentam melhores resultados.
    \item A variação do período de correção não tem grande influência nos resultados.
\end{enumerate}

Relativamente ao ponto 2), os relógios das raspberry Pis são bastante parecidos e apresentam drifts muito baixos (cerca de $15ppm$, segundo \cite{b4}). Contudo, um drift de $15ppm$ ao logo de uma hora representa um erro de $36 ms$, e ao longo de uma semana de $6 s$, o que já é critico para um sistema de semáforos. Ora, apesar de os resultados das experiências com sincronização serem piores a curto prazo, este são constantes com o tempo, ou seja, ao longo de uma semana o erro seria sensivelmente o mesmo nos casos com sincronização. 

Os relógios sincronizados apresentam um desempenho inferior a curto prazo, devido ao o ajuste constante do rate, que causa variações que dependem das condições da rede. Por exemplo, na figura \ref{fig:delay_vs_erro} em volta dos $80 minutos$, as condições da rede variam muito o que resulta num aumento do erro quadráticos da precisão das slots. Este efeito é minimizado pela consideração do delay no cálculo do rate. A figura \ref{fig:rate_vs_delay_caso1} apresenta um segmento em que houve uma variação do delay que não foi considerada, o que resulta numa proporcional variação do rate. Enquanto que na figura \ref{fig:rate_vs_delay_caso2} esta variação foi considerada, resultando num rate mais estável. Estas figuras, revelam também a diferença entre as equações \ref{eq:rate_sem_delay} e \ref{eq:rate_com_delay}, respetivamente.

Relativamente ao ponto 3), evidenciam-se que os resultados da experiência com correção de rate no servidor S1 e no servidor S3 com correção de delay, offset e rate. Na verdade, o servidor S1 é bastante instável: para além de apresentar um jitter elevado (1 segundo) tem também uma disponibilidade muito baixa. Ora, nesta experiência o servidor esteve indisponível durante vários minutos, sendo que nos instantes anteriores o delay variou bastante, comprometendo o rate. Assim os relógio utilizaram os últimos parâmetros disponíveis para a sua sincronização, que estavam incorretos, resultando em erros na precisão das slots de $10 s$ em poucos minutos. Em contraste a este resultado, a experiência no servidor 3 com correção de offset, rate e delay teve um jitter de $26 ms$ e um erro na precisão das slots de $8 ms$, o baixo jitter conduziu a baixa variação de rate, que durante toda a experiência apenas variou $0.0088$, o que é muito baixo comparando com a outra experiência nesse servidor que com um jitter de $0.7229$ o rate variou $0.7590$.

No que diz respeito ao ponto 3, como os relógios das raspberry Pi apresentam drifts muito baixos, as correções não precisam de ter um período muito baixo, pelo que os resultados são idênticos para vário períodos de sincronização. Esta conclusão é fundamentada pelo bom desempenho da experiência sem correção.

\subsection{Conclusões finais}

Os resultados deste projeto corroboram a influência do jitter na sincronização de relógio. Para além disso, permitem inferir sobre a melhor metodologia de sincronização. Apesar de os melhores resultados serem com correção de rate, delay e offset, os sistema distribuídos atuais não usam o offset como parâmetro de correção, visto que pode resultar em ajustes negativos o que leva a que o mesmo relógio apresente o mesmo timestamp em dois momentos consecutivos.

Para além disso, uma introdução do delay no cálculo do rate não é uma metodologia usual na sincronização. De facto, sistema atuais utilizam vários valores passados de offset e períodos de atualização segundo o servidor NTP para inferir o rate, enquanto que neste projeto apenas foi utilizado a amostra anterior do período segundo o servidor NTP.

Por fim, o resultado das sincronizações seria mais evidente caso a duração dos teste fosse maior. Deste modo, seriam necessário vários dias de teste para pelos menos duas metodologias: sem sincronização e com sincronização.

\subsection{Contribuições}

O grupo que realizou o projeto é constituído por dois elementos, sendo as contribuições as seguintes:

\begin{outline}
    \1 André de Azevedo Barata - 50\%
        \2 Implementação do cliente;
        \2 Métodos de sincronização;
        \2 Geração dos resultados;
        \2 Análise dos resultados;
   
    \1 Diogo Vilela - 50\%
        \2 Métodos de sincronização;
        \2 Montagem do sistema;
        \2 Realização das experiências;
        \2 Análise dos resultados;
\end{outline}








\begin{thebibliography}{00}
\bibitem{b1} Digi-Key Electronics, https://www.digikey.pt/pt/products/detail/raspberry-pi/SC0194-9/10258781

\bibitem{b2} D.L. Mills. Internet time synchronization: the network time protocol. IEEE Transactions on Communications, 39(10):1482–1493, 1991.



\end{thebibliography}


\end{document}
